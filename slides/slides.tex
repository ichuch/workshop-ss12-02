%% LaTeX-Beamer template for KIT design
%% by Erik Burger, Christian Hammer
%% title picture by Klaus Krogmann
%%
%% version 2.1
%%
%% mostly compatible to KIT corporate design v2.0
%% http://intranet.kit.edu/gestaltungsrichtlinien.php
%%
%% Problems, bugs and comments to
%% burger@kit.edu

\documentclass[18pt]{beamer}

\usepackage[utf8]{inputenc}
\usepackage[babel,german=quotes]{csquotes}
\usepackage{graphicx}
\usepackage{caption}
\usepackage{subfig}
\usepackage[right]{eurosym}
\usepackage{listings}


\usepackage{amssymb} % for |N
\newcommand{\Hilight}{\makebox[0pt][l]{\color{cyan}\rule[-4pt]{0.65\linewidth}{14pt}}}	%from http://newsgroups.derkeiler.com/Archive/Comp/comp.text.tex/2008-10/msg00877.html
\lstset{tabsize=3}

\newcommand{\refframe}
{
	\usebackgroundtemplate
	{
		\begin{pgfpicture}{0mm}{0mm}{\paperwidth}{\paperheight}
		
			\pgfsetnonzerorule
			
			% draw bg color
			{
				\color{black!15}
				\pgfpathrectangle{\pgfpoint{0mm}{0mm}}{\pgfpoint{\paperwidth}{\paperheight}}
				\pgfusepath{fill}
			}
			
			%+ draw grey frame
			{
				\pgfsetcornersarced{\pgfpoint{2mm}{2mm}}
				
				\pgfpathmoveto{\pgfpoint{\paperwidth-1mm}{9mm}}
				\pgfpathlineto{\pgfpoint{1mm}{9mm}}
				\pgfpathlineto{\pgfpoint{1mm}{\paperheight-1mm}}
			}

			{
				\pgfsetcornersarced{\pgfpoint{2mm}{2mm}}
				\pgfpathmoveto{\pgfpoint{1mm}{\paperheight-1mm}}
				\pgfpathlineto{\pgfpoint{\paperwidth-1mm}{\paperheight-1mm}}
				\pgfpathlineto{\pgfpoint{\paperwidth-1mm}{9mm}}
			}
			%- draw grey frame

			\color{yellow}
			\pgfusepath{fill}

		\end{pgfpicture}
	}
}



\author{Robert Schneider, Sven Brauch}


\institute{}

\begin{document}

% change the following line to "ngerman" for German style date and logos
\selectlanguage{ngerman}

\AtBeginSection[]{%
	\begin{frame}
		\tableofcontents[sectionstyle=show/hide,subsectionstyle=hide/show/hide]
	\end{frame}
	\addtocounter{framenumber}{-1}% If you don't want them to affect the slide number
}

%title page
\begin{frame}
\titlepage
\end{frame}

%table of contents
\begin{frame}{Gliederung}
\tableofcontents
\end{frame}


\section{Objektorientierte Programmierung}
\begin{frame}
    \frametitle{Structures}
    \begin{block}{structs}
    Ein \texttt{struct} ist eine Zusammenfassung mehrerer Objekte zu einem größeren.
    Zum Beispiel könnte man ein \texttt{struct} "`Quader"' erstellen, welches drei Fließkommazahlen beinhaltet.
    \end{block}
    \begin{block}{Instanzen}
    Eine solche \texttt{struct} ist lediglich eine abstrakte Beschreibung des Objekts; man arbeitet schließlich mit sogenannten \emph{Instanzen} des Objekts. Beispiel Quader: Die Structure an sich beschreibt das abstrakte Objekt, die Instanz einen konkreten Quader ("`der Quader auf meinem Tisch"').
    \end{block}
\end{frame}
\begin{frame}
    \frametitle{Beispiel für eine Structure}
    \vspace{0.7cm}
    \includegraphics[width=15cm]{example_code/box.pdf}
\end{frame}

\begin{frame}
    \frametitle{Klassen}
    \begin{block}{Klassen}
    Eine \texttt{class} ist eine \texttt{struct}, die zusätzlich zu Daten noch Funktionen enthält, die auf diesen Daten operieren.
    \end{block}
    \begin{block}{Konstruktor und Destruktor}
    Eine Klasse hat zwei besondere Funktionen, den Konstruktor und den Destruktor; der Konstruktor wird aufgerufen, wenn eine neue Instanz der Klasse erstellt wird, und der Destruktor, wenn die Instanz wieder gelöscht wird. Der Konstruktor heißt \texttt{klassenname}, der Destruktor \texttt{\~{}klassenname}.
    \end{block}
\end{frame}
\begin{frame}
    \frametitle{Beispiel für eine Klasse}
    \vspace{0.7cm}
    \includegraphics[width=8.7cm]{example_code/box2.pdf}
\end{frame}
\begin{frame}
    \frametitle{Etwas anderes Beispiel für eine Klasse}
    \includegraphics[width=10.0cm]{example_code/box3.pdf}
\end{frame}


%%%%%%%%%%%%%%%%%%%%%%%%%
% ADD OWN SECTIONS HERE %
%%%%%%%%%%%%%%%%%%%%%%%%%
\include{cpp}
\include{things}

\section{Stack und Heap}
\include{stack}
\include{heap}


\end{document}
